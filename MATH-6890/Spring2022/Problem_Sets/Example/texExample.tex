\documentclass[11pt]{article}

\usepackage{graphicx}
\usepackage{amsmath}
\usepackage{amssymb}
\usepackage{color}

\usepackage{listings}
\lstset{
basicstyle=\footnotesize\ttfamily,
columns=flexible,
breaklines=true,
commentstyle=\color{red},
keywordstyle=\color{black}\bfseries}

\textwidth=6.5in
\textheight=9in
\topmargin=-0.8in
\headheight=15.75pt
\headsep=.35in
\oddsidemargin=0.0in
\evensidemargin=0.0in


\begin{document}
\begin{flushright}
\small{MATH-6840\\
J. W. Banks}
\end{flushright}

\begin{center}
\large{Example of using \LaTeX ~for problem set solutions}\\
\end{center}

\begin{enumerate}
  \item {\color{blue} \em I would often have the original question restated here in italics.}\\
  
  Then my response would appear here. Assume in this case that I was asked about a heat equation. In \LaTeX ~the equation can be nicely displayed like
\begin{align*}
  u_t(x,t) & =\nu u_{xx}(x,t)\\
  u(x,0) & = u_0(x).
\end{align*}


\item {\color{blue} \em We might also have need to submit plots and code. For example, suppose we were asked to plot $\sinh(x)$ and $\cosh(x)$ for $x\in[-\pi,\pi]$.}\\

The requested plot is given in Figure~\ref{fig:example}, and the MATLAB code to produce this is given in Listing~\ref{code:myCode}.
\begin{figure}[hbt]
  \begin{center}
  \includegraphics[width=2in]{images/plotExample}
  \caption{Example plot of $\sinh(x)$ and $\cosh(x)$.}
  \label{fig:example}
  \end{center}
\end{figure}

\lstinputlisting[language=Matlab, numbers=left, stepnumber=1, firstline=1,caption={Example code},label=code:myCode,frame=single]{codeExample.m}

\end{enumerate}

\end{document}
